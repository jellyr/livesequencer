\documentclass[a4paper]{easychair}
\title{Haskell Live Programming}
\titlerunning{Haskell Live Programming}
\author{
  Henning Thielemann \\ Universität Halle 
  \and
  Johannes Waldmann \\ HTWK Leipzig
}
\authorrunning{Thielemann and Waldmann}

\begin{document}
\maketitle

\begin{abstract}
  We present the design of a software system 
  that allows to interleave the execution of Haskell code
  with user interactions that change parts of the code.
  The system presents visial aids for the programmer,
  to see ``what's currently happening''.
  The system is used for ``live composition'' of music,
  where a potentially infinite stream of MIDI events is produced.
  Another application is visualization of the Haskell
  evaluation strategy, which is useful in  teaching or debugging.
\end{abstract}

\section{Introduction}

\paragraph{Motivation from a musician's perspective.}

A piece of music can be represented as a sequence of events
(note-on, note-off) with parameters (instrument, pitch, etc.)
The MIDI standard \cite{mma:midispec} defines such a notation.

Such a stream of events is obtained as the rendering
of a composition that has a tree-like structure,
where the leaf nodes are basic events,
and the inner nodes are operators (e.g., for
sequential and parallel composition, and transformations).
The Haskore language \cite{hudak:theory} defines such a tree-like structure
(an algebra).

In a live performance, the musician typically wants to change
the tree model, in order to include spontaneous ideas
(improvisations) of his own, or in reaction to actions of
the other musicians in a group performance.
Such improvised changes usually are localized: Some
sub-tree is changed while the context of the subtree is kept.

The performance must be kept going, so an edit/compile/run mode
of operation cannot be used. This poses the challenge of
rendering and editing a structured composition at the same time.

\paragraph{Motivation from a programmer's perspective.}

When trying to understand the observable behaviour of a computer program,
it can be helpful to inspect some of the internal behaviour
of the (virtual) machine that executes the program.
E.g., in debugging sessions for programs in an imperative language,
the ``current statement'' is highlighted, and ``current assignment''
of some variables that are in scope is shown.

For the execution of Haskell programs, 
a similar tracing of program execution is desirable.
Haskell can be considere a rule-based language:
a program consists of function definitions,
these are ordered equations that are applied
(as rewrite rules) to the ``main'' expression that is under evaluation.

Therefore the execution of a Haskel program is a sequence of states,
where a state is the expression being rewritten,
and a state transition is the application of one program rule
to one subterm (redex).

The idea of live programming can then be modelled by
adding the rules to the state, and introducing another kind of
state transitions, namely, changes in the rules.

\paragraph{Lazy Evaluation and live programming.}

Haskell's lazy evalution fits well with music,
and with live programming:

A music performance is a stream of events.
These events happen in time, and future events need not be
computed too much in advance. It is enought to produce events as needed.
This allows potentially infinite compositions
that still could be rendered in finite space. 
(Quite often, the space is bounded as the composition is a finite automaton).

Now it is a natural idea that changes in the program
will influence the computation of future events
This is obtained for free since for each state,
the ``tail'' of the event stream is un-evaluated (since Cons is lazy),
and when evaluation happens, it will use the then-current rules.

\paragraph{Design considerations.}

The input language syntactically is a subset of Haskell,
and for this subset, the semantics should be identical to that of Haskell.

Reasons: programmers don't have to learn a different language,
programs could be executed (with the same output, but without live editing) 
on any Haskell system.


\section{Realization of Operational Semantics.}

To simplify matters for the moment, we consider first-order programs only.
The underlying model is term rewriting,
with an evaluation strategy that corresponds 
to Call-By-Need evaluation where the ``need'' arised by pattern matching.

A rewrite step consists of 
\begin{itemize}
\item 
  determining a position (a subterm) in the input term,
  according to the evaluation strategy
\item
  determining a substitution,
  by matching the left-hand side of a rule
  with the subterm,
\item 
  the application of this substitution
  to the right-hand side of this rule.
\end{itemize}

\section{Named Subterms}

Assume a rule $l \to f(A,B)$ is applied to the input expression,
for a function symbol $f$, and sub-terms $A,B$.
Assume $f$ has the semantics of ``append''.
Evaluation proceeds, and the subterm $A$ is evaluated
(the sub-composition is rendered).

What happens if the programmer changes the expression $B$
in the source text? Actually, nothing will change, 
since the original shape of $B$ is already present
in the current term, so the sub-composition $B$ will
be rendered as it was when the rule was originally given.
This may or may not be the intented semantics.

We can transform the program: add a fresh symbol $b$,
and consider the rule set $\{l \to f(A, b), b\to B\}$ instead.
The overall semantics is identical (with one extra
reduction step $b\to B$ that would go unnoticed since it does not
procuce events)
but the live-programming behaviour is different:
as long as $A$ is evaluated, the rule $b\to B$
could be replaced by another rule $b\to B'$
and this change would have an effect.

The transformation, as given, does not handle the case
that $B$ contains variables. 
Values for these variables will be determined by matching $l$, 
but they need to be transported. 
Assume $l$ and $B$ contain a variable $v$,
then the transformed program should be $\{l \to f(A,b(v)), b(v) \to B\}$.

It is conceivable to apply this transformation
to each subterm of the right-hand side of each rule.
(In other words, each subterm gets a fresh name,
and is made a ``top-level'' entity.)

The execution of the program then involves additional steps
that correspond to the distribution of the ``current substitution''
down the right-hand side. This is typical for calculi
of explicit substitution \cite{DBLP:conf/popl/Lescanne94}.

The program transformation should be hidden from the programmer.
Instead, the source code parser needs to ensure a reasonable relation
between the names for subterms, from one version of the program text
to the next, using some form of common-subtree detection.

A welcome side effect of this translation is that
each reduction step refers to one subterm (of the rhs),
so this position can be highlighted in the editor window,
so the programmer can observer the execution of the program.

\section{Implementation}

Implementation started October 8, 2011.
The software is written in Haskell,
using Alsa/Midi modules,
and the WX graphical user interface toolkit.

\begin{itemize}
\item download source code: \url{git://dfa.imn.htwk-leipzig.de/srv/git/seq/}
\item browse source code: \url{http://dfa.imn.htwk-leipzig.de/cgi-bin/gitweb.cgi?p=seq.git}
\item bugs database: \url{http://dfa.imn.htwk-leipzig.de/bugzilla/describecomponents.cgi?product=live-sequencer}
\end{itemize}

\section{Related Work}

\begin{itemize}
\item systems for live programming of music
\item hot code replacement in Erlang
\item Java code replacement in debugging sessions in Eclipse
\end{itemize}

\bibliographystyle{alpha}
\bibliography{seq}

\end{document}
