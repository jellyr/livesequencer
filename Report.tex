\begin{hcarentry}{Live-Sequencer}
\report{Henning Thielemann}
\status{experimental, active}
\participants{Johannes Waldmann}% optional
\makeheader

The Live-Sequencer allows to program music in the style of Haskore,
but it is inherently interactive.
You cannot only listen to changes to the music quickly,
but you can alter the music while it is played.
Changes to the music may not have an immediate effect
but are respected when their time has come.

Additionally users can alter parts of the modules of a musical work
via a WWW interface.
This way multiple people including the auditory
can take part in a live composition.
This mode can also be used in education,
when students shall solve small problems in an exercise.

Technical background:
The music is represented as lazy list of MIDI events.
(MIDI is the Musical Instrument Digital Interface).
The MIDI events are sent via ALSA and
thus can control any kind of MIDI application,
be it software synthesizers on the same computer
or external hardware synthesizers.
The application can also receive MIDI events
that are turned into program code.
We need certain ALSA functionality for precise timing of events.
Thus the sequencer is currently bound to Linux.

The Live-Sequencer can be run either as command-line program
without editing functions
or as an interactive program based on wxwidgets.

The used language is a much simplified kind of Haskell.
It provides no sharing, misses many syntactic constructs and is untyped.
However the intersection between Haskell and the Live-Sequencer language
is large enough for algorithmic music patterns
and we provide several examples that are contained in this intersection.


\FuturePlans
\begin{compactitem}
\item Define proper semantics for live changes to a program
\item Use of Helium's parser, module system and type checker
\item Refined reduction steps for educational purposes
\item Highlighting of active terms that better fits to the music
\end{compactitem}

\FurtherReading
Development information, videos and papers can be found at
\begin{center}
  \url{http://www.haskell.org/haskellwiki/Live-Sequencer}
\end{center}
\end{hcarentry}
